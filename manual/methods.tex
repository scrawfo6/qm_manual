\chapter{Quantum Monte Carlo Methods}
\label{chap:qmcmethods}

\begin{table}[h]
\begin{center}
\begin{tabularx}{\linewidth}{l l l l l X }
\hline
\multicolumn{6}{l}{\texttt{qmc} factory element} \\
\hline
\multicolumn{2}{l}{parent elements:} & \multicolumn{4}{l}{\texttt{simulation, loop}}\\
\multicolumn{2}{l}{type   selector:} & \multicolumn{4}{l}{\texttt{method} attribute}\\
\multicolumn{2}{l}{type   options: } & vmc           & \multicolumn{3}{l}{Variational Monte Carlo}\\
%\multicolumn{2}{l}{                } & opt           & \multicolumn{3}{l}{}\\
\multicolumn{2}{l}{                } & linear        & \multicolumn{3}{l}{Wavefunction optimization with linear method}\\
%\multicolumn{2}{l}{                } & cslinear      & \multicolumn{3}{l}{}\\
\multicolumn{2}{l}{                } & dmc           & \multicolumn{3}{l}{Diffusion Monte Carlo}\\
\multicolumn{2}{l}{                } & rmc           & \multicolumn{3}{l}{Reptation Monte Carlo}\\
%\multicolumn{2}{l}{                } & ptcl          & \multicolumn{3}{l}{}\\
%\multicolumn{2}{l}{                } & mul           & \multicolumn{3}{l}{}\\
%\multicolumn{2}{l}{                } & warp          & \multicolumn{3}{l}{}\\
\multicolumn{2}{l}{shared attributes:} & \multicolumn{4}{l}{}\\
   &   \bfseries name         & \bfseries datatype & \bfseries values & \bfseries default & \bfseries description \\
   &   \texttt{method}        &  text              &   listed above   & invalid           & QMC driver            \\
   &   \texttt{move}          &  text              &   pbyp, alle     & pbyp              & Method used to move electrons \\
   &   \texttt{gpu}           &  text              &   yes, no        & dep.              & Use the GPU\\
   &   \texttt{trace}         &  text              &                  & no                & ???                      \\
   &   \texttt{checkpoint}   &  integer           &   -1, 0, n       & -1                & Checkpoint frequency \\
   &   \texttt{record}      &  integer           &   n              & 0                & Save configuration every n steps  \\
   &   \texttt{target}        &  text              &                  &                   & ???  \\
   &   \texttt{completed}     &  text              &                  &                   & ???  \\
   &   \texttt{append}        &  text              &   yes, no        & no               & ???  \\
%   &   \texttt{multiple}      &  text              &   yes, no        & no                & ???  \\
%   &   \texttt{warp}          &  text              &   yes, no        & no                & ???  \\
\hline

\end{tabularx}
\end{center}
\end{table}

Additional information:
\begin{itemize}
\item \texttt{move}: There are two ways to move electrons. The more used method is the particle-by-particle move. In this method, only one electron is moved for acceptance or rejection. The other method is the all-electron move; namely, all the electrons are moved once for testing acceptance or rejection.

\item \texttt{gpu}: When the executable is compiled with CUDA, the target computing device can be chosen by this switch. With a regular CPU-only compilation, this option is not effective.

\item \texttt{checkpoint}:
This enables and disables checkpointing and specifying the frequency of output.  Possible values are:
\begin{description}
\item [-1] No checkpoint (default setting).
\item [0] Dump after the completion of a QMC section.
\item [n] Dump after every $n$ blocks.  Also dump at the end of the run.
\end{description}
%\item  ``-1'' No checkpoint (default setting).
%\item ``0'' Dump after the completion of a qmc section.
%\item ``n'' Dump after every $n$ blocks.  Also dump at the end of the run.
%\end{itemize}
%If Checkpoint=``-1'' no checkpoint will be done (default setting). If Checkpoint=``0'' dump after the completion of a qmc section. When dumconfig=``n'' is present with Checkpoint=``0'', the configurations will be dumped at the end of the run (due to Checkpoint=``0''), but also at every n block. If Checkpoint=``n'', configurations will be dump every n block. 

% TODO: Fill in more information about checkpoint/restart

The particle configurations are written to a \texttt{.config.h5} file.

%Other sections of data needed for
%restart include the random generator state, written to a \texttt{.random.xml} file.




% I think this description refers to a previous version of the config.h5 file.  See HDFWalkerInput0.cpp.
%All the dumped data will be written in a \texttt{*.config.h5} file. The \texttt{config.h5} file will contain the state of a population to continue a run. The list of what is included in the .config.h5 is: number of walkers, status of the run, branch mode, energy dataset, ratio to accepted moves, ratio to proposed moves, variance dataset, vParam\{tau, taueff. E\_trial, E\_ref, Branch\_Max, BranchCutOff, BranchFilter, Sigma, Accepted\_Energy, Accepted\_Samples\}, IParam{warmumSteps, Energy\_Update\_Interval, Counter, targetwalkers, Maxwalkers, MinWalkers, Branching Interval}, Walker coordinates, Random number size, Random number sequence, version of the code.

\begin{lstlisting}[style=QMCPXML,caption=The following is an example of running a simulation that can be restarted. ]
  <qmc method="dmc" move="pbyp"  checkpoint="0">
    <parameter name="timestep">         0.004  </parameter>
    <parameter name="blocks">           100   </parameter>
    <parameter name="steps">            400    </parameter>
  </qmc>
\end{lstlisting}

The checkpoint flag instructs QMCPACK to output walker configurations.  This also
works in VMC.  This outputs an h5 file with the name \texttt{projectid.run-number.config.h5}.
Check that this file exists before attempting a restart.

To continue a run, specify the \texttt{mcwalkerset} element before your VMC/DMC block:
\begin{lstlisting}[style=QMCPXML,caption=Restart (read walkers from previous run) ]
 <mcwalkerset fileroot="BH.s002" version="0 6" collected="yes"/>
  <qmc method="dmc" move="pbyp"  checkpoint="0">
    <parameter name="timestep">         0.004  </parameter>
    <parameter name="blocks">           100   </parameter>
    <parameter name="steps">            400    </parameter>
  </qmc>
\end{lstlisting}
\texttt{BH} is the project id, and \texttt{s002} is the calculation number to read in the walkers from the previous run.

In the project id section, make sure that the series number is different from any existing ones to avoid overwriting them. 

\end{itemize}
